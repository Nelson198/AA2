% Setup -------------------------------

\documentclass[a4paper]{article}
\setcounter{secnumdepth}{3}
\setcounter{tocdepth}{3}

\PassOptionsToPackage{hyphens}{url}
\usepackage{hyperref}

% Encoding
%--------------------------------------
\usepackage[T1]{fontenc}
\usepackage[utf8]{inputenc}
%--------------------------------------

% Portuguese-specific commands
%--------------------------------------
\usepackage[portuguese]{babel}
%--------------------------------------

% Hyphenation rules
%--------------------------------------
\usepackage{hyphenat}
%--------------------------------------

% Capa do relatório

\title{
	Aprendizagem Automática II
	\\ \Large{\textbf{Proposta para o Trabalho Prático}}
	\\ -
	\\ Mestrado em Engenharia Informática
	\\ \large{Universidade do Minho}
}

\author{
	\begin{tabular}{ll}
		\textbf{Grupo}
        \\\hline
        PG41080 & João Ribeiro Imperadeiro
        \\
		PG41081 & José Alberto Martins Boticas
		\\
        PG41091 & Nelson José Dias Teixeira
        \\
        PG41851 & Rui Miguel da Costa Meira
	\end{tabular}
}

\date{\today}

\begin{document}

\maketitle

\section{Introdução}
\normalsize{
    Tal como foi requerido pelo docente desta disciplina, este documento serve para fazer um 
    levantamento preliminar do que será desenvolvido para o respetivo trabalho prático. Exibe-se
    de seguida o repositório \textit{GitHub} público associado a este projeto onde será colocado todo
    o código desenvolvido pelo grupo bem como toda a documentação e recursos intrínsecos à implementação:
    \begin{center}
        \textit{\url{https://github.com/Nelson198/AA2}}
    \end{center}
}

\section{Tema}
\normalsize{
    ...
}

\section{Planificação}
\normalsize{
    ...
}

\section{Objetivos}
\normalsize{
    \begin{enumerate}
        \item Garantir que ...
        \item Verificar que 
    \end{enumerate}
}

\end{document}