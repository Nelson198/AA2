% Setup -------------------------------

\documentclass[a4paper]{article}
\setcounter{secnumdepth}{3}
\setcounter{tocdepth}{3}

\PassOptionsToPackage{hyphens}{url}
\usepackage{hyperref}
\usepackage{indentfirst}

% Encoding
%--------------------------------------
\usepackage[T1]{fontenc}
\usepackage[utf8]{inputenc}
%--------------------------------------

% Portuguese-specific commands
%--------------------------------------
\usepackage[portuguese]{babel}
%--------------------------------------

% Hyphenation rules
%--------------------------------------
\usepackage{hyphenat}
%--------------------------------------

% Capa do relatório

\title{
	Aprendizagem Automática II
	\\ \Large{\textbf{Proposta para o Trabalho Prático}}
	\\ -
	\\ Mestrado em Engenharia Informática
	\\ \large{Universidade do Minho}
}

\author{
	\begin{tabular}{ll}
		\textbf{Grupo}
        \\
        \hline
        PG41080 & João Ribeiro Imperadeiro
        \\
		PG41081 & José Alberto Martins Boticas
		\\
        PG41091 & Nelson José Dias Teixeira
        \\
        PG41851 & Rui Miguel da Costa Meira
	\end{tabular}
}

\date{\today}

\begin{document}

\maketitle

\section{Introdução}
\normalsize{
    Tal como foi requerido pelo docente da unidade curricular \textsl{Aprendizagem Automática II}, este documento serve para fazer um 
    levantamento preliminar do que será desenvolvido para o trabalho prático da respetiva UC. Exibe-se,
    de seguida, o repositório \textit{GitHub} público associado a este projeto, onde será colocado todo
    o código desenvolvido pelo grupo bem como toda a documentação e recursos intrínsecos à implementação:
    \begin{center}
        \textit{\url{https://github.com/Nelson198/AA2}}
    \end{center}
}

\section{Tema}
\normalsize{
    O tema escolhido pelo nosso grupo é o desenvolvimento de uma \textit{framework} de \textsl{AutoML}.
    A \textit{framework} visa obter o melhor modelo para problemas de \textit{supervised learning} e \textit{unsupervised learning}, 
    de forma automática e com a menor intervenção possível por parte do programador.
    O objetivo final é colocar a \textit{framework} disponivel para os utilizadores da linguagem \textit{Python}.
}

\section{Planificação}
\normalsize{
    Devido à complexidade do desenvolvimento deste projeto e atendendo ao curto espaço de tempo disponível, não serão incluídas opções de pré-processamento de dados.
    Com isto, o utilizador/programador deverá indicar qual o tipo de modelo (regressão, classificação ou \textit{clustering}) que deseja obter, 
    sendo depois da responsabilidade da \textit{framework} a procura do melhor modelo desse tipo, visitando todos os algoritmos disponíveis.
    Se o utilizador preferir um algoritmo em especial poderá indicá-lo, sendo da responsabilidade da \textit{framework} a procura dos melhores hiperparâmetros.
    
    A \textit{framework} vai distinguir entre um problema de \textit{supervised} e \textit{unsupervised learning} consoante receba, ou não, as \textit{labels}/\textit{targets}.
    Todos os problemas de \textit{supervised learning} vão ser distinguidos entre regressão e classificação, dependendo se o \textit{target} é uma variável contínua ou discreta.
    
    Para problemas de regressão, os algoritmos disponiveis serão: 
    \begin{itemize}
        \item Regressão Linear;
        \item Regressão Polinomial;
        \item \textit{Support Vector Regression};
        \item \textit{Decision Tree Regression};
        \item \textit{Random Forest Regression};
        \item Redes Neuronais.
    \end{itemize}
    
    Para problemas de classificação, os algoritmos disponiveis serão:
    \begin{itemize}
        \item Regressão Logística;
        \item \textit{k-Nearest Neighbors} (\textsl{KNN});
        \item \textit{Support Vector Machine} (\textsl{SVM});
        \item \textit{Kernel} \textsl{SVM};
        \item \textit{Naive Bayes};
        \item \textit{Decision Tree Classification};
        \item \textit{Random Forest};
        \item Redes Neuronais.
    \end{itemize}
    
    Para problemas de \textit{clustering}, os algoritmos disponiveis serão:
    \begin{itemize}
        \item \textit{k-Means Clustering}
        \item \textit{Hierarchical Clustering}
    \end{itemize}
    
    A procura de melhores hiperparâmetros para Redes Neuronais vai ser realizada com o uso da biblioteca \texttt{kerastuner}. 
    Para outros algoritmos, vai-se fazer uso do \textit{GridSearchCV} e \textit{RandomizedSearchCV} para esta escolha.

    Na eventualidade da plataforma estar terminada e ainda existir tempo para tal, será estudada a possibilidade de se incluir o pré-processamento de dados, 
    aumentando não só a complexidade como também a flexibilidade da \textit{framework}. Isto poderá permitir que sejam testados modelos de tipos distintos.
}

\section{Objetivos}
\normalsize{
    \begin{enumerate}
        \item Permitir o teste de diferentes modelos, com diferentes algoritmos, para um certo conjunto de dados;
        \item Comparar diferentes modelos, apresentando as suas métricas;
        \item Encontrar o melhor modelo, com base nas métricas apresentadas;
        \item Disponibilizar a plataforma \textit{online}, para uso da comunidade;
        \item Preparar o projeto para a inclusão de pré-processamento de dados.
    \end{enumerate}
}

\end{document}
